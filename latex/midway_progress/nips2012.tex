\documentclass{article} % For LaTeX2e
\usepackage{nips12submit_e,times}


\usepackage{listings}
\definecolor{javared}{rgb}{0.6,0,0} % for strings
\definecolor{javagreen}{rgb}{0.25,0.5,0.35} % comments
\definecolor{javapurple}{rgb}{0.5,0,0.35} % keywords
\definecolor{javadocblue}{rgb}{0.25,0.35,0.75} % javadoc
% "define" Scala
\lstdefinelanguage{scala}{morekeywords={class,object,trait,extends,with,new,if,while,for,def,val,var,this},
otherkeywords={->,=>},
sensitive=true,
morecomment=[l]{//},
morecomment=[s]{/*}{*/},
morestring=[b]"}
% Default settings for code listings
\lstset{frame=tb,language=scala,aboveskip=3mm,belowskip=3mm,showstringspaces=false,columns=flexible,basicstyle={\small\ttfamily},
keywordstyle=\color{javapurple}\bfseries,
stringstyle=\color{javared},
commentstyle=\color{javagreen},
morecomment=[s][\color{javadocblue}]{/**}{*/}}
%\documentstyle[nips12submit_09,times,art10]{article} % For LaTeX 2.09


\title{Formatting Instructions for NIPS 2012}


\author{
David A. Leen \\
Department of Applied Mathematics\\
University of Washington \\
\texttt{dleen@uw.edu} \\
\And
Brian D. Walker \\
Computer Science \& Engineering \\
University of Washington \\
\texttt{walker7734@gmail.com} \\
}

% The \author macro works with any number of authors. There are two commands
% used to separate the names and addresses of multiple authors: \And and \AND.
%
% Using \And between authors leaves it to \LaTeX{} to determine where to break
% the lines. Using \AND forces a linebreak at that point. So, if \LaTeX{}
% puts 3 of 4 authors names on the first line, and the last on the second
% line, try using \AND instead of \And before the third author name.

\newcommand{\fix}{\marginpar{FIX}}
\newcommand{\new}{\marginpar{NEW}}

\nipsfinalcopy % Uncomment for camera-ready version

\begin{document}
\maketitle

\begin{abstract}
{\em Milestone progress:} We implement the Hogwild! algorithm and demonstrate an approximately linear speedup over the 
sequential stochastic gradient descent (SGD) algorithm.  We use the click prediction dataset and its sparse search token features to achieve a near optimal rate of convergence.
\end{abstract}

\section{Introduction}

\section{Method}

\subsection{Proof of concept}
A fundamental example of the concept in Scala is:
\begin{lstlisting}
val m = collection.mutable.Map[Int, Double]() // create mapping
for (i <- (0 until 10)) m.put(i, 0.0) // initialize key values to 0.0
for (i <- (0 until 20).par) m(i % 10) += 1.0 // update 
\end{lstlisting}
For the non-parallel version of this program we expect a mapping  \verb+0->2.0+, \verb+1->2.0+,\ldots. This parallel version produces an output like \verb+0->1.0+, \verb+1->2.0+, \verb+1->1.0+\ldots. Where in some cases the threads have overwritten each other (hopefully not many in the full case).

This is the concept that we will use.
\subsection{Algorithm}

\section{Challenges and issues}
\label{gen_inst}

\subsection{Parallel and concurrent programming}

\subsection{Storing data in memory}

\subsection{Non-sparse data}

\section{Results}
\label{headings}

\subsection{Near linear speedup}

Second level headings are lower case (except for first word and proper nouns),
flush left, bold and in point size 10. One line space before the second level
heading and 1/2~line space after the second level heading.

\section{Further work}
Do not change any aspects of the formatting parameters in the style files.
In particular, do not modify the width or length of the rectangle the text
should fit into, and do not change font sizes (except perhaps in the
\textbf{References} section; see below). Please note that pages should be
numbered.


\subsubsection*{References}
\cite{niu2011hogwild}
\cite{bradley2011parallel}
\cite{zhang2012comunication}

\begingroup
\renewcommand{\section}[2]{}%
%\renewcommand{\chapter}[2]{}% for other classes
\bibliographystyle{unsrt}
\bibliography{references}	
\endgroup


\end{document}
